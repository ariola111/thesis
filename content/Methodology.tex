\chapter{Methodology}

\section{Implementation}


\subsection{Components And Libraries Used}

- Golden layout component
- version 3 - written with ts
- Support for vueJS
- had a hard time setting up, but I made it work
setup: 
installed vue-golden-layout
created a Composable Hook to make golden layout work for the current 

- Add a header 
- Add a sidebar container for the sl-vue-tree
adding the sidebar component 


- Creating the file explorer component
- considering quasar framework
- Not sure if its the right choice for the current setup
- based on VueJS , open-source
- state of the art 
- capacitor docs 
- gerat framework 
- cross-platform native runtime that makes it easy to build modern web apps
- successor to Apache Cordova and Adobe PhoneGap
- Oriented to cross platform apps including mobile apps (not really the case for this application)
- 
- I need a tree viewer instead of a file explorer - Checking out tree viewer
libraries 
- [sl-vue-tree] [vue-awesome] 





-\section{Feature List}
- Outline Feature - When activated a popup window shows up with the complete diagram Also shows which part of the diagram you are viewing
- Layers 
- Zoom (zoom in zoom out , detail zooming) 
- Hide/ show 
- Undo redo 
- Delete 
- Positioning (To Front to back) 
- Connections /Waypoints 

- Cognitive Load of learning 
- scroll through the toolbar 
- Favorites toolbar 

- Explorer 
 	- different views on the side 
 	- git debugger project management file explorer 
- Conversion 
- Different versions of the same object Usually power of 2 ( Preview in levels) 
- Visualize hierarchy in depth show it in layers Layer you focus on 
 
 Display of information 
 	- desktop / tablet mode 
- Styling 
	- fill color , stroke color, shadow
	- filled color background with a border
	- specific rules are fixed for the user
	- principles of material design (use this concepts for the design)
	- Editor theme
	-Different types of edges
	- How do we draw connectors
		- if its on screen (drag and drop)
		- if its not on the viwepoint 
	Join edge tool -> drag and drop
	Pop up with the text input (unique identifiers for each)
	
- Editing space (Canvas) fixed or infinite canvas 
- Serialization 
- Parallel States / different files for each chart  
- opening multiple files at the same time 
- Create different features based on the fact we have diff files - Combine 
- Copy paste feature 
- Relation between file and canvas 
- Right click on the file 
	- copy file 
	- copy diagram  
	- Paste in editor  
	- paste in context  
	- Delete 
- select  
	- press delete 
	- open console 
	- ctrl+c 
	
- shortcuts 
- 

- Clear interafce 
- Color codes for different parts to select the context 
- Architecture to implement/ create support new tools 
- group tools into collections 
	

-\subsection{Implemented}

- Layout tool ( golden layout library ) 


