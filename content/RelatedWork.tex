\chapter{Related Work}

\section{Concepts}

\subsection{State Charts}
-what are UML state charts (Citing from 2002 integrated formal methods book << Contributions for Modelling UML State-Charts in B)
start citation << 
State chars are diagrams are UML diagrams for modelling dynamic aspects of systems. UML state-chars focus on the event-ordered behaviour of an object, a feature which is specially useful in modelling a reactive system   >> end citation

State of the art (Citing from 2002 integrated formal methods book  << Contributions for Modelling UML State-Charts in B )

This section recalls essential points in the work of Meyer, Nguyen and Lano for modelling
UML state-charts in B. Those works are based on the UML state-chart concepts inherited
from OMT state-chart [21]: state, sequential sub state, concurrent sub state, transition,
action and non deferred event. Each of such elements is modelled by a derivation scheme.
The B derivation of UML state-charts is integrated in the B derivation of class diagrams.

-- Check out the paper below referenced from the citation above
J. Rumbaugh, M. Blaha, W. Premerlani, F. Eddy, and W. Lorensen. Object-Oriented Mod-
eling and Design. Prentice Hall Inc. Englewood Cliffs, 1991.


-- Cool paper about model driven engineering (Citing from 2002 integrated formal methods book  << Model driven engineering )

<< start citing  --  Model Driven Engineering (MDE) is wider
in scope than MDA. MDE combines process and analysis with archi-
tecture. end citing >>




\section{Similar Works}

\subsection{Subsection One}

\subsection{Subsection Two}


\section{Algorithms used}
	
\section{Where does my thesis stand}



